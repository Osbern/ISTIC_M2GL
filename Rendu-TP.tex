\documentclass{article}

\usepackage[utf8]{inputenc}
\usepackage[T1]{fontenc}
\usepackage[french]{babel}
\usepackage{tabularx}

\title{Module --
\\
V \& V
\\
--}
\author{Thibaud Destouches - Lacroix Marceau}
\date{2012 - 2013}

\begin{document}

\begin{titlepage}
\maketitle
\tableofcontents
\end{titlepage}

\newpage
\section{Introduction}

Le but de ce TP est d’aborder le test d’Interface Homme-Machine. Pour ce faire nous avons été ammenés à utiliser le plugin eclipse windowTester Pro. Le projet que nous devons tester est un logiciel de gestion de contacts.

\section{Tests effectués}

\subsection{Tests demandés}
\begin{enumerate}
\item All text fields should be disabled at application startup.
\item All text fields should be enabled after clicking the new button.
\item The button for delete should be enabled when there's a selected item in the list.
\item The Save button should not be enabled right after clicking the New button (since there's no data in the fields). 
\item The button for Delete should not be enabled where there isn't a selected item in the list.
\end{enumerate}
\subsection{Tests supplémentaires}
\begin{itemize}
\item Lors de la saisie des données, entrer un '@' dans l'un des champs de texte fait planter l'application.	
\item Lorsqu'aucun élément n'est selectionné, Edit et Delete sont désactivés.
\item Lors d'une édition, si aucun des champs n'est mofifié, il est impossible de sauvegarder.
\end{itemize}

\section{Améliorations \& Corrections}
Nous avons apporté de nombreuses modifications au niveau du comportement des boutons (actif/inactif) en fonction de l'état de l'application.
\begin{itemize}
\item Par défaut (pas de séléction d'une fiche dans la liste) seul le bouton new est activé; lors d'une selection Edit et Delete s'activent.
\item Cliquer sur Delete active les champs pour édition.
\item New affiche une nouvelle fiche éditable; si tous les champs sont vide (ou s'ils sont remplis puis vidés) la sauvegarde n'est pas possible. 
\item Les boutons de suppression et d'édition sont désactivés lors de la saisie d'un contact.
\item Le bouton Save est désactivé si aucun champ n'est modifié lors d'une édition.
\end{itemize}
\newpage


\end{document}
